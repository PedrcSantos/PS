\chapter{Criação e edição de necessidades}\label{cap:necessityCreation}

O utilizador irá ser redirecionado para este ecrã sempre que cria ou altera uma necessidade.
A funcionalidade deste ecrã é criar necessidades ou alterar necessidades existentes.
Quando um utilizador clica no botão \textit{Create Necessity} virá para este ecrã com o intuito de criar uma necessidade.
Neste caso, o ecrã apresentará os vários campos necessários para a criação desta.
Se o utilizador desejar editar uma necessidade, primeiramente, só o poderá fazer se for o autor desta.
Neste caso, o ecrã apresentará os campos já preenchidos com os dados atuais da necessidade e a possibilidade de os alterar livremente.

Os botões presentes neste ecrã são apenas dois:
\begin{enumerate}
    \item \textit{Cancel} - aprensenta um \textit{pop-up} a pedir a confirmação do utilizador, após confirmação a criação ou edição será cancelada e o utilizador redirecionado para o ecrã onde estava anteriormente.
    \item \textit{Save} - guarda as alterações a efetuar na necessidade e cria uma ligação ao servidor para modificar a base de dados com uma nova necessidade ou alterando uma necessidade pré-existente.
\end{enumerate}
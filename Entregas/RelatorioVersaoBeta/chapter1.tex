\chapter{Introdução}\label{sec:intro}

\section{Enquadramento}\label{sec:enquadramento}
O mercado de hoje, cada vez mais tecnológico, exigente e desafiador, impõe um ritmo às empresas que, para
além de gerirem os seus principais processos de negócio, estas têm também uma dinâmica significativa de
atividades internas para as ajudar no seu crescimento e competitividade. Em particular, as empresas do setor
das tecnologias de informação, mantêm atividades internas tais como participação em feiras de emprego,
partilha de conhecimento através de apresentações informais, desenvolvimento de componentes de software,
ofertas de formação, atividades lúdicas, entre muitas outras. Para isso, tem que existir uma coordenação de
recursos que nem sempre é fácil, dada a sua alocação aos projetos em curso. Contudo, se existir um
planeamento atempado gerido através de uma plataforma de colaboração, o processo pode ser agilizado,
permitindo não só o registo e divulgação das necessidades internas, bem como a aceitação de candidaturas por
parte dos colaboradores mais interessados na sua realização.

\section{Objetivos}\label{sec:objectivos}
A aplicação proposta visa a implementação de uma plataforma colaborativa para agilizar a resposta a necessidades internas das empresas, nomeadamente a
organização de eventos, partilha de conhecimento, ofertas formativas, entre outras. 
Os objetivos que a plataforma visa atingir consistem na divulgação de comunicados para a comunidade da empresa;
no contexto de um evento, incluem a partilha de localização, 
gestão de recursos, registo de presenças, registo de candidaturas e gestão das mesmas; 
a calendarização de eventos e, por último, um sistema de notificações.
% Classe do documento e parâmetros gerais.
\documentclass[a4paper,openright,twoside,11pt]{report}


%
% Times New Roman font.
%
\usefont{T1}{ptm}{m}{n}
\selectfont


% Packages a utilizar e respetivos parâmetros.
\usepackage{inputenc}
\usepackage[portuguese]{babel}
\addto{\captionsportuguese}{\renewcommand{\bibname}{Refer\^{e}ncias}}
\addto{\captionsportuguese}{\renewcommand{\contentsname}{\'Indice}}
\addto{\captionsportuguese}{\renewcommand{\appendixname}{Anexo}}
\usepackage{graphicx}
\usepackage{url}
\usepackage[Algoritmo]{algorithm}
\usepackage{algorithmicx}
\usepackage{algpseudocode}
\renewcommand{\algorithmicrequire}{\textbf{Dados: }}
\renewcommand{\algorithmicensure}{\textbf{Resultado: }}

% Definições das dimensões das páginas
\setlength{\textheight}{24.00cm}
\setlength{\textwidth}{15.50cm}
\setlength{\topmargin}{0.35cm}
\setlength{\headheight}{0cm}
\setlength{\headsep}{0cm}
\setlength{\oddsidemargin}{0.25cm}
\setlength{\evensidemargin}{0.25cm}

%
% Times New Roman font.
%
\usefont{T1}{ptm}{m}{n}
\selectfont


%\renewcommand{\baselinestretch}{1}

%
% Página inicial (capa)
%
\title{
   \vspace{-60mm}
   \begin{minipage}[l]{160mm}
      \resizebox{50mm}{!}{\includegraphics{../img/logo_isel.png}}\\
   \end{minipage}\\
   \vspace{20mm}
		% Título do projeto na forma capitalizada. A primeira letra de cada palavra deve ser maiúscula.
   {\bf *T\'{i}tulo do Trabalho de Projecto \\ Usando a Primeira Letra na Forma Maiúscula*}
}

% Nome dos autores (um por linha)
\author{
\begin{tabular}{ll}
             & *Fernando Pessoa*  \\
             & *Ricardo Reis* 
\end{tabular}}


\date{
\vspace{80mm}
\begin{tabular}{ll}
  {Orientadores} & *Álvaro de Campos* \\
                  & *Alberto Caeiro, SoftCompany*\\
\end{tabular}\\
% Deixar o indicador respetivo em função da versão do relatório.
\vspace{10mm}
Relatório *de progresso* *beta* final* realizado no âmbito de Projecto e Seminário,\\
do curso de licenciatura em Engenharia Informática e de Computadores\\
Semestre de Verão 2014/2015
\vspace{20mm}
*Maio* de 2015}


\begin{document}

\pagenumbering{roman}
\thispagestyle{empty}
\maketitle

\baselineskip 18pt % line spacing: 12pt for single, 18pt for 1 1/2, and 24pt for double spacing

\newpage
\thispagestyle{empty}
% Fim da contracapa

% Página com identificação completa (número e nome) e assinaturas dos estudante(s) e do(s) orientador(es)
%\cleardoublepage\newpage
%\setcounter{page}{1}
%\begin{center}
%{\Large\bf Instituto Superior de Engenharia de Lisboa}\\
%{\large Licenciatura em Engenharia Informática e de Computadores}\\
%%Projecto e Seminário\\
%\vspace{50mm}
%{\large \bf  *T\'{i}tulo do Trabalho de Projecto \\ Usando a Primeira Letra na Forma Maiúscula*}\\
%\vspace{20mm}
%\begin{tabular}{rl}
%  *75463* & *Fernando António Nogueira Pessoa*\\
%  **72453* & *Ricardo Manuel Augusto dos Santos Reis*\\
%\end{tabular}\\
%\vspace{8mm}
%\noindent\rule{12cm}{0.6pt}\\
%\vspace{10mm}
%\noindent\rule{12cm}{0.6pt}\\
%\vspace{10mm}
%\begin{tabular}{rl}
%  Orientadores: & *Álvaro José Silva de Campos*\\   
%                & *Alberto Joaquim Alves Caeiro, SoftCompany*\\
%\end{tabular}\\
%\vspace{8mm}
%\noindent\rule{12cm}{0.6pt}\\
%\vspace{10mm}
%\noindent\rule{12cm}{0.6pt}\\
%\vspace{15mm}
%Relatório *de progresso* *beta* final* realizado no âmbito de Projecto e Seminário,\\
%do curso de licenciatura em Engenharia Informática e de Computadores\\
%Semestre de Verão 2014/2015\\
%\vspace{20mm}
%*Maio* de 2015\\
%\end{center}

% Página de resumo em Português
\cleardoublepage\newpage
\chapter*{Resumo}
Texto do resumo.

Breve descrição do projecto, dos resultados importantes e das conclusões: o objectivo é dar ao leitor uma visão global do projecto (não deve exceder uma página). 

{\bf Palavras-chave:} lista de palavras-chave, ordenadas alfabeticamente, separadas por ;.

%% Página de resumo em Inglês
%\cleardoublepage\newpage
%\chapter*{Abstract}
%Abstract text (1 page).\\
%
%{\bf Keywords:} sorted keyword list, delimited by ;.

%% Página de agradecimentos
%\cleardoublepage\newpage
%\chapter*{Agradecimentos}
%Texto dos agradecimentos. é opcional.\\

% Geração do índice de conteúdos
\cleardoublepage\newpage
\tableofcontents
\cleardoublepage

% Geração do índice de figuras
\listoffigures
\cleardoublepage

% Geração do índice de tabelas
\listoftables
\cleardoublepage

% Iniciar a numeração de páginas
\setcounter{page}{1}
\pagenumbering{arabic}

% Capitulo 1
%\include{capitulo1}

% Capitulo 2
%%
% Cap�tulo 2
%
\chapter{Formula��o do Problema} \label{cap2}

Estamos no in�cio do novo cap�tulo. Aqui podemos colocar algum texto introdut�rio e 
de resumo do conte�do do cap�tulo. Por exemplo, a sec��o~\ref{sec21} trata aspectos
referentes �s cita��es de bibliografia. Na sec��o~\ref{sec22} apresenta-se um exemplo 
de enumera��o de conte�dos. O uso de tabelas � exemplificado na sec��o~\ref{sec23}.
Nas sec��es~\ref{sec24} e~\ref{sec25} abordam-se express�es matem�ticas e o uso de
figuras de grandes dimens�es.


%
% 2.1 - Nome da sec��o
%
\section{Nome da sec��o} \label{sec21}
Agora o texto da sec��o. Em \cite{wiki:bigdata2015} encontra v�rias refer�ncias para o assunto. Segue-se a explica��o das refer�ncias \cite{Boytsov:2011:IMA:1963190.1963191} e \cite{Jurkiewicz:2015:MVA:2627368.2656337}. Exemplos de livros da �rea s�o \cite{Neumann:1958:CB:578873} e \cite{Kernighan:1982:EPS:578130}.

Este segundo par�grafo � a continua��o da sec��o.


%
% 2.2 - An�lise do problema
%
\section{An�lise do problema - enumera��o} \label{sec22}
Nesta an�lise vamos considerar uma vers�o simplificada do problema de apresenta��o de listas 
de enumera��o. A unidade curricular Projecto e Semin�rio do curso de licenciatura em Engenharia Inform�tica e de
Computadores proporciona a oportunidade para demonstrar independ�ncia e originalidade, para planear e
organizar um projecto durante um per�odo de tempo limitado, e para p�r em pr�tica t�cnicas ensinadas ao
longo do curso. O semin�rio, em articula��o com o projecto, destina-se � introdu��o de temas relevantes para
os estudantes.

Projecto e Semin�rio tem dura��o semestral, envolvendo, em m�dia, tr�s dias de trabalho semanais do
estudante, ao longo de 20 semanas, a que correspondem 18 cr�ditos ECTS (480 horas de trabalho do
estudante). No final, o estudante:
\begin{itemize}
	\item [1.] Planeou, executou e completou um projecto e, de forma apropriada, implementou-o no per�odo de
tempo previsto;
  \item [2.] Utilizou o orientador, apropriadamente, como consultor do projecto ou como cliente.
  \item [3.] Fez duas comunica��es (das quais uma no �mbito do projecto) e arguiu uma.
  \item [4.] Demonstrou compet�ncia pr�tica e os resultados do projecto.
  \item [5.] Documentou o projecto, designadamente no relat�rio final.
\end{itemize}

%
% 2.3 - Outro problema - tabela
%
\section{Outro problema - tabela} \label{sec23}
Em muitas situa��es, � necess�rio e conveniente apresentar os resultados na
forma de tabela. Assim, a tabela~\ref{tab1} apresenta os prazos de entrega de Projecto
e Semin�rio, para o semestre de Ver�o 2014/2015.

\begin{table} [h!]
\centering
\caption{Um exemplo de legenda de tabela. Prazos de entrega de Projecto
e Semin�rio, para o semestre de Ver�o 2014/2015.} \vspace{2mm}
\label{tab1}       % Dar um nome de etiqueta �nico.
\begin{tabular}{|l|l|l|}
\hline
\textbf{Data} & \textbf{Actividade} & \textbf{Observa��es}  \\ \hline
23 de Mar�o de 2015 & Proposta do projecto & Quatro p�ginas \\ \hline
4 de Maio de 2015 & Relat�rio de progresso & Preparar bem\\ \hline
									& Apresenta��o individual & Escolher tema \\ \hline
15 de Junho de 2015 & Cartaz e vers�o beta & \\ \hline
25 de Julho de 2015 & Vers�o final (�poca normal) & \\ \hline
19 de Setembro de 2015 & Vers�o final (�poca especial) & � necess�ria inscri��o\\ \hline
\end{tabular}
\end{table}


%
% 2.4 - Express�es matem�ticas
%
\section{Express�es matem�ticas} \label{sec24}
As express�es matem�ticas tais como $a= b + c = d/e$ s�o necess�rias em muitas situa��es.
Podemos ter express�es n�o numeradas, tal como na linha anterior, ou ainda desta forma
\begin{displaymath} 
h = \sqrt{a^2 + b^2},
\end{displaymath}
e podemos ter express�es numeradas tais como
\begin{equation} \label{eq1}
E = m c^2,
\end{equation}
as quais s�o elementos do texto e podem ser referidas pela sua etiqueta (n�mero) 
da seguinte forma atrav�s de~(\ref{eq1}), � semelhan�a do que acontece para 
figuras e tabelas.\\

As express�es podem envolver fun��es conhecidas, tais como
\begin{equation} \label{eq2}
\displaystyle s(t)=\sum_{i=1}^{N} a_i x_i(t) =a_1 x_1(t) +  a_2 x_2(t) + \ldots + a_N x_N(t) \qquad \mbox{e} \qquad \displaystyle p(t)=\sum_{i=1}^{N} \log_2(a_i) x_i(t).
\end{equation}


%
% 2.5 - Figuras de grande dimens�o
%
\section{Figuras de grande dimens�o} \label{sec25}
Por vezes, em casos excepcionais devido � sua dimens�o, as figuras t�m 
que ser apresentadas de forma a ocupar toda a p�gina, na forma de paisagem
(\emph{landscape}). Podemos fazer isso da forma que se apresenta na figura~\ref{fig:logotipo2}.


% Colocar uma figura
\begin{figure}[h]
\begin{center}
\includegraphics[scale=0.99,angle=90]{./figures/logoISEL.png}
\end{center}
\caption{Legenda da figura com o logotipo do ISEL - vers�o 2.}
\label{fig:logotipo2}
\end{figure}
	

% Capitulo 3
%%
% Cap�tulo 3
%
\chapter{Solu��o Proposta - Grandes Ideias} \label{cap3}

A nossa solu��o � apresentada neste cap�tulo. A solu��o consiste em grandes 
ideias, desenvolvidas e testadas.

Exemplo de indenta��o do segundo par�grafo.


%
% Sec��o 3.1
%
\section{Nome da primeira sec��o deste cap�tulo} \label{sec31}
Texto da sec��o. Seguem-se exemplos de v�rios par�grafos.

Esta unidade curricular funciona no semestre de Ver�o de cada ano lectivo. Nos casos de impedimento
prolongado justificado (designadamente por doen�a ou por motivos profissionais no caso dos
trabalhadores-estudantes), poder� ser prolongada, havendo lugar � elabora��o de outro relat�rio de progresso
e a nova inscri��o se o prolongamento for al�m do per�odo de �poca especial desse semestre. A entrega da
justifica��o e a sua aprecia��o dever�o ocorrer antes do final do prazo estabelecido para a entrega final.\\

O estudante s� poder� frequentar Projecto e Semin�rio se, em conjunto com as restantes unidades
curriculares em que se inscreve nesse semestre isso corresponder, no m�ximo, a 42 cr�ditos ECTS, tendo
acumulado, pelo menos, 138 cr�ditos. No caso de estudantes em regime de tempo parcial, o valor m�ximo
est� limitado a 30 cr�ditos no ano lectivo. N�o s�o admitidas inscri��es como unidade curricular isolada.\\

Anualmente � divulgada a lista de ideias para projectos e respectivos orientadores. Os estudantes poder�o
propor outras ideias identificando os orientadores. A escolha da ideia de projecto � feita no per�odo de
interrup��o lectiva ap�s o semestre de Inverno. As propostas de projecto s�o registadas no in�cio do per�odo
lectivo do semestre de Ver�o, verificado que os estudantes re�nem as condi��es de frequ�ncia.
O projecto deve ser realizado em grupo de dois estudantes (excepcionalmente um ou tr�s). Cada elemento do
grupo tem tarefas espec�ficas pelas quais � respons�vel. Esta situa��o deve ficar clara desde o in�cio do
projecto.\\

A orienta��o dos projectos � feita por docentes da �rea departamental onde o curso est� ancorado ou por
especialistas externos, podendo haver co-orientadores, mas sendo obrigat�ria a co-orienta��o por docente da
�rea departamental no caso de orienta��o externa. O desenvolvimento do projecto � acompanhado de
reuni�es peri�dicas do orientador (e/ou co-orientadores) com o grupo. A informa��o referente ao projecto �
mantida em formato electr�nico em local acess�vel pelos elementos do grupo, pelos orientadores e pelos
docentes de Projecto e Semin�rio.\\

A avalia��o de Projecto e Semin�rio envolve: 
\begin{enumerate}
	\item proposta do projecto; 
	\item relat�rio de progresso; 
	\item apresenta��o individual; 
	\item cartaz e vers�o beta do projecto; 
	\item relat�rio de projecto e discuss�o p�blica final.
\end{enumerate}
A avalia��o incide sobre o trabalho planeado e desenvolvido pelos estudantes, com constri��es de tempo e prazos
previamente estabelecidos. Se durante a realiza��o do projecto for considerado que este est� em risco,
ouvidos os estudantes envolvidos, o orientador e o docente da unidade curricular decidem se o projecto
continua. Em caso de desist�ncia do estudante, esta deve ser comunicada ao orientador do projecto e ao
regente da unidade curricular.


%
% Sec��o 3.2
%
\section{A segunda sec��o deste cap�tulo} \label{sec32}
Na segunda sec��o deste cap�tulo, vamos abordar o enquadramento, 
o contexto e as funcionalidades.

%
% Sec��o 3.2.1
%
\subsection{A primeira sub-sec��o desta sec��o} \label{sec321}
As sub-sec��es s�o �teis para mostrar determinados conte�dos de forma
organizada. Contudo, o seu uso excessivo tamb�m n�o contribui para a facilidade
de leitura do documento.

%
% Sec��o 3.2.2
%
\subsection{A segunda sub-sec��o desta sec��o} \label{sec322}
Esta � a segunda sub-sec��o desta sec��o, a qual termina aqui.


%
% Sec��o 3.3
%
\section{Descri��o detalhada da solu��o} \label{sec33}
A solu��o proposta assenta nas seguintes ideias. O algoritmo~\ref{alg1}
apresenta as ac��es de pesquisa de um elemento $E$ sobre um grafo $G$.
\begin{algorithm}
\caption{Algoritmo de pesquisa em grafo.}
\label{alg1}
\algorithmicrequire{Grafo G, Elemento E}\\
\algorithmicensure{Localiza��o de E em G}\\
\begin{enumerate}
\item Para todos os v�rtices $v$ em $G$
\item Pesquisar e obter a localiza��o de $E$
\begin{enumerate}
	\item Iniciar a lista de pontos, $P$
	\item Ordenar $P$
\end{enumerate}	
\end{enumerate}
\end{algorithm} 

\newpage
Nalgumas situa��es, � necess�rio apresentar alguns tro�os de 
c�digo que ilustrem determinados aspectos relevantes da implementa��o.

\begin{verbatim}
namespace ps;
public static void main() {
		System.out.println(``PS - Projecto e Semin�rio'');
}
\end{verbatim}





% Capitulo 4
%%
% Cap�tulo 4
%
\chapter{Avalia��o Experimental} \label{cap4}

A avalia��o da nossa solu��o � apresentada neste cap�tulo. Aqui mostramos
como as nossas grandes ideias funcionaram 

Exemplo de indenta��o do segundo par�grafo.


%
% Sec��o 4.1
%
\section{Nome da primeira sec��o deste cap�tulo} \label{sec41}
Texto da sec��o. 

Continua��o do texto noutro par�grafo.


%
% Sec��o 4.2
%
\section{A segunda sec��o deste cap�tulo} \label{sec42}
Na segunda sec��o deste cap�tulo, vamos abordar o enquadramento, 
o contexto e as funcionalidades.

%
% Sec��o 4.2.1
%
\subsection{A primeira sub-sec��o desta sec��o} \label{sec421}
As sub-sec��es s�o �teis para mostrar determinados conte�dos de forma
organizada. Contudo, o seu uso excessivo tamb�m n�o contribui para a facilidade
de leitura do documento\footnote{Este � um exemplo de nota de rodap�. Devem ser usadas com modera��o, pois retiram a aten��o ao texto principal.}.

%
% Sec��o 4.2.2
%
\subsection{A segunda sub-sec��o desta sec��o} \label{sec422}
Esta � a segunda sub-sec��o desta sec��o, a qual termina aqui.


%
% Sec��o 4.3
%
\section{An�lise de resultados} \label{sec43}
A an�lise de resultados segue aqui, nos pr�ximos par�grafos de forma detalhada.

A data limite de entrega da vers�o final em 19 de Setembro de 2015 tem subjacente a inscri��o em �poca
especial. N�o se verificando esta situa��o, a data limite de entrega � em 25 de Julho de 2015.
O j�ri de cada projecto � constitu�do por indica��o do respectivo orientador, at� 25 de Maio de 2015. A
avalia��o da vers�o beta ser� realizada at� 29 de Junho de 2015.\\

A discuss�o da vers�o final do projecto � p�blica e inclui at� 30 minutos de apresenta��o e demonstra��o
seguidos de discuss�o at� ao m�ximo de 120 minutos de dura��o total da prova (40 e 150 minutos,
respectivamente, quando o grupo tem tr�s estudantes, e, 20 e 90 minutos quando o trabalho � individual). O
j�ri da prova � proposto pelo orientador do projecto, tem pelo menos tr�s elementos e inclui o arguente, o
orientador e um docente de Projecto e Semin�rio (podendo este, em caso de impedimento, delegar num
docente da �rea departamental). As delibera��es do j�ri s�o tomadas por maioria simples.\\

A melhoria de classifica��o, se realizada no semestre de Inverno seguinte, ter� por base trabalho adicional e
discuss�o oral avaliados pelo mesmo j�ri. Quando realizada noutro semestre, envolve a realiza��o de novo
projecto.\\

A coordena��o global dos projectos e dos semin�rios � feita pelos docentes de Projecto e Semin�rio, de
acordo com as orienta��es definidas pela comiss�o coordenadora do curso. No s�tio desta unidade curricular,
� mantida a informa��o relevante, incluindo prazos, calend�rio dos semin�rios, estado dos projectos e
estudantes e orientadores envolvidos. No final de cada ano lectivo, o regente elabora e apresenta � comiss�o
coordenadora do curso um relat�rio sucinto sobre o funcionamento da unidade curricular. Em cada ano
lectivo s�o identificados os melhores projectos para promover a sua divulga��o p�blica.



% Capitulo 5
%%
% Cap�tulo 5
%
\chapter{Conclus�es} \label{cap5}
Neste trabalho tratou-se o problema. Foi formulada a solu��o 
que assenta nos princ�pios de boas pr�ticas aprendidos ao longo do curso.

A solu��o obtida atingiu resultados satisfat�rios.

% Referências
\bibliographystyle{unsrt}
\bibliography{referencias}

% Anexos (opcional)
\appendix

% Anexo 1
%%
% Anexo 1
%
\chapter{Diagramas da Aplica��o} \label{a1}

Estamos no in�cio do anexo 1. Nalguns casos, � conveniente colocar anexos de forma
a complementar os resultados. Por vezes, em casos excepcionais devido � sua dimens�o, as figuras t�m 
que ser apresentadas de forma a ocupar toda a p�gina, na forma de paisagem
(\emph{landscape}). Podemos fazer isso da forma que se apresenta na figura~\ref{fig:a11}.
% Colocar uma figura
\begin{figure}[h]
\begin{center}
\includegraphics[scale=0.99,angle=90]{./figures/logoISEL.png}
\end{center}
\caption{Diagrama de casos de utiliza��o.}
\label{fig:a11}
\end{figure}
	

% Anexo 2
%%
% Anexo 2
%
\chapter{Modelos de dados} \label{a2}

Estamos no início do anexo 2.

O relat�rio � um resumo do projecto global. Apenas como refer�ncia, � expect�vel cerca de 30 a 40 p�ginas A4 n�o devendo exceder 50 p�ginas. A estrutura deve ser discutida e aceite pelo orientador. Os cap�tulos apresentados devem ter, em geral, a seguinte organiza��o:\\

\textbf{Cap�tulo 1} - Introdu��o\\
Enquadramento do trabalho, metas, objectivos e especifica��es do projecto e resumo da solu��o. Concluir com a descri��o breve dos restantes cap�tulos.

\textbf{Cap�tulo 2} - Formula��o do problema\\
Introdu��o dos conhecimentos necess�rios para entendimento do trabalho, estabelecimento de terminologia e descri��o detalhada do problema e do seu contexto. S�ntese de abordagens anteriores do problema, caso existam, indicando as raz�es porque s�o insatisfat�rias. 

\textbf{Cap�tulo 3} - Grande ideia 1\\
Este cap�tulo pode ser subdividido em sec��es, designadamente:\\
1.	Introdu��o: descri��o da abordagem do problema e da metodologia utilizada; identifica��o das tarefas;\\
2.	Elenco das caracter�sticas / An�lise do problema: especifica��es, constri��es, ferramenta de an�lise, etc.\\
3.	Projecto: modelo para resolu��o do problema;\\
4.	Implementa��o: a implementa��o do modelo como sistema computacional; descri��o concisa do hardware e do software; dificuldades e contradi��es encontradas e sua resolu��o;\\
5.	Avalia��o: testes realizados e resultados experimentais (quando poss�vel, o objectivo, a montagem e o m�todo usado devem ser brevemente descritos); an�lise cr�tica dos resultados.\\

\textbf{Cap�tulo k+1} - Grande ideia k\\

\textbf{Cap�tulo k+3} - Conclus�es\\
Recapitula��o do trabalho desenvolvido. Referir claramente as observa��es e conclus�es importantes. Discuss�o de ideias e recomenda��es para trabalho futuro.

\textbf{Refer�ncias}
Elenco dos livros e artigos citados no relat�rio. As refer�ncias s�o numeradas consecutivamente ao longo do relat�rio. O n�mero da refer�ncia deve estar entre par�ntesis rectos: [1].

\textbf{Anexos}\\
Os anexos devem incluir as partes importantes do dossier do projecto. O seu conte�do depende da natureza do projecto, mas, em geral, pode incluir: listagem de programas, resultados de testes de software, exemplos de ecr�s de interface com o utilizador, esquemas dos circuitos, listagem de componentes, data sheets cr�ticos, resultados de testes de hardware, desenhos mec�nicos, an�lise econ�mica, etc. (quando realiz�vel, o relat�rio deve ser acompanhado da c�pia do c�digo, bibliotecas, etc. em suporte electr�nico).

\textbf{Mais algumas dicas}\\
O j�ri para avalia��o do projecto final de curso apreciar� o projecto, a sua demonstra��o e o respectivo relat�rio final (valorizando a escrita enquanto forma de divulga��o de conhecimento). O relat�rio, depois de aceite e discutido, ficar� dispon�vel na biblioteca do departamento, para consulta.\\

O relat�rio deve ser auto-suficiente, isto �, o professor ou qualquer aluno finalista deve ficar apto a perceber o trabalho que realizou sem ter de ir � biblioteca ler os artigos originais.\\

N�o escreva para o orientador, conhecedor de todo o detalhe, ou para um principiante. Tente escrever para uma audi�ncia constitu�da por estudantes finalistas. Mantenha em mente o n�vel de conhecimentos do leitor a que se dirige. O relat�rio ser� uma base de trabalho para estudantes em circunst�ncias semelhantes. N�o dificulte o trabalho do leitor nem o fa�a est�pido (obviamente,...). Tamb�m � imposs�vel ser totalmente claro. Evite afirma��es dogm�ticas (exemplo: "O software � a parte mais importante do computador.").\\

O relat�rio t�cnico n�o � uma hist�ria: usualmente n�o segue a cronologia das actividades realizadas. Tamb�m n�o � um romance (aten��o � adjectiva��o). O relat�rio � um documento formal, feito para descrever os aspectos importantes do trabalho realizado.\\

N�o tente descrever a fun��o de cada componente, por exemplo a frase "O circuito IC2 e os componentes a ele associados formam um amplificador inversor..." � adequada. Contudo, descreva detalhadamente a fun��o de cada componente ou circuito invulgar ou cr�tico.\\

As ilustra��es (figuras, tabelas, gr�ficos e exemplos) s�o auxiliares preciosos para a explica��o, mas envolvem muito trabalho. As figuras e as tabelas devem ser leg�veis, instrutivas, legendadas e ter t�tulo. Os exemplos devem ser suficientemente detalhados para ilustrar o conceito.
O texto deve, pelo menos, ser analisado por um corrector ortogr�fico: os erros de ortografia s�o inadmiss�veis.\\

Recomenda-se a leitura de alguns artigos e ou livros bem escritos para adquirir sensibilidade para a arte de escrever. Os artigos premiados em confer�ncia s�o, normalmente, bons exemplos de escrita.\\

A escrita do relat�rio demora sempre mais tempo do que o inicialmente previsto.\\

No essencial, a ideia � que tem algo para vender e o ``Resumo'' � a montra: a mensagem deve ser suficientemente clara e encorajar o cliente a entrar - se ele n�o a perceber passar� ao lado. O resumo inclui: a motiva��o para o trabalho, como o fez e os resultados principais. Devem ser evitados chav�es e palavras longas, as refer�ncias s�o proibidas e n�o deve utilizar acr�nimos. Tenha em conta que o leitor ser� influenciado quer pela informa��o contida no resumo quer pela maneira como este est� escrito. N�o h� desculpas para frases curtas ou desligadas, erros de gram�tica ou erros de sintaxe.\\

N�o � f�cil escrever um bom resumo.\\

\textbf{Introdu��o}\\
Procure dar resposta �s seguintes quest�es: qual � o problema? porque � importante? o que � que outros j� fizeram? quais as ideias base da solu��o apresentada? como est� organizado o resto do relat�rio?

\textbf{Formula��o do problema}\\
Defina o problema. Introduza a terminologia. Discuta as propriedades b�sicas.\\

\textbf{Solu��o do problema}\\
Enumere as coisas que fez e que considere importantes. N�o seja modesto mas tamb�m n�o exagere.
A correcta avalia��o do projecto � um aspecto cr�tico.\\

\textbf{Conclus�es}\\
Procure dar resposta �s seguintes quest�es: quais, se for caso disso, as li��es aprendidas? o que, se algo, foi explicado? em que medida os objectivos foram atingidos? se existe algo que agora faria de forma diferente? quais as vantagens e desvantagens do trabalho realizado face a exemplos da literatura? que ideias para trabalho futuro?\\

\textbf{Refer�ncias}\\
A ideia subjacente � refer�ncia � que esta poupa papel e que o leitor poder� obter o documento em qualquer biblioteca cient�fica razo�vel. Assim, � crit�rio essencial referir revistas dispon�veis em bibliotecas de institui��es de ensino superior ou outras institui��es profissionais. Em geral, n�o � razo�vel a refer�ncia a actas de confer�ncias, dado que estas raramente est�o acess�veis em bibliotecas pelo que, para todos os efeitos, est�o indispon�veis. As refer�ncias a ``Comunica��es Privadas'' s�o inaceit�veis. A informa��o dada deve ser sempre suficientemente detalhada por forma a que o leitor possa adquirir a publica��o ou consult�-la numa biblioteca. Refer�ncias a disserta��es de doutoramento ou outras devem indicar a institui��o e o seu endere�o. Sendo a refer�ncia essencial para o trabalho, no caso desta ser dif�cil de obter, dever-se-� incluir no documento, ou em ap�ndice, os seus pontos essenciais.
Cite uma refer�ncia sempre que est� a incluir algo que n�o � contribui��o sua ou quer indicar um conjunto de refer�ncias que o leitor pode consultar, mas cujo conte�do n�o pode ser descrito adequadamente no relat�rio.


\end{document}
